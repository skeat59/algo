 \documentclass [11pt]{report}

\usepackage{fancyhdr}
\usepackage [french]{babel}

\usepackage[utf8]{inputenc}
\usepackage[T1]{fontenc}
\usepackage{textcomp}
\usepackage{graphicx}
\usepackage{titlepic}

\usepackage{listings}
\usepackage{minitoc}
\usepackage{footmisc}
\usepackage{color}
\usepackage{graphicx}

\usepackage{tikz}
\usetikzlibrary{arrows}


\usepackage{eso-pic}
\pagestyle{fancy}	

\tikzset{
  treenode/.style = {align=center, inner sep=0pt, text centered,
    font=\sffamily},
 arn_r/.style = {treenode, circle, black, draw=black, 
     text width=1.5em, very thick},
  arn_r/.style = {treenode, circle, black, draw=black, 
    text width=1.5em, very thick},
  arn_x/.style = {treenode, rectangle, draw=black,
    minimum width=0.5em, minimum height=0.5em}% arbre rouge noir, nil
}


\begin{document}
\underline{{\huge Exercice 1}}\\
\indent \underline{question 1:} \vspace{2.5mm}
	 repr\'esentation par tas de la figure 1.\\
\begin{tabular}{|c|c|c|c|c|c|c|c|c|c|}
\hline 5 & 8 & 12 & 9 & 11 & 20 & 15 & 18 & 10 & 13 \\ 
\hline 
\end{tabular} 

\vspace{5mm}

\indent \underline{question 2:}\\
a- La racine se trouve a place 1 du vecteur.\\
b- Les fils d'un nœud se trouve \`a 2x(place du nœud) et \`a 2x(position du nœud) + 1 \\
c- Pour retrouver le p\`ere d'un noeud il faut faire (position du nœud) div 2. O\`u div est la division enti\`ere\\
d- Le nœud est une feuille dans le cas ou les cases du vecteur 2x(position du nœud) et 2x(position du nœud) + 1 sont vide ou si elle n'existe pas\\
e- Un nœud est un point simple si l'une des cases, correspondant a ses fils, dans le vecteur est vide.\\

\vspace{10mm}

\underline{{\huge Exercice 2:}}\\
\indent \underline{question 1: Ajout}\\
\indent a- On ajoute l'\'el\'ement en feuille, puis on le fais remonter en l'\'echangeant avec son p\`ere jusqu'a ce qu'il se trouve a sa place. c'est-\`a-dire lorsque son p\`ere est inf\'erieur au nouvel \'el\'ement.\\
\indent b- Apr\`es l'ajout de la valeur 4:

\begin{tikzpicture}[->,>=stealth',level/.style={sibling distance = 5cm/#1,
  level distance = 1.5cm}] 
\node [arn_r] {4}
    child{ node [arn_r] {5} 
            child{ node [arn_r] {9} 
            	child{ node [arn_r] {18}} 
				child{ node [arn_r] {10}}
            }
            child{ node [arn_r] {8}
							child{ node [arn_r] {13}}
							child{ node [arn_r] {11}}
            }                            
    }
    child{ node [arn_r] {12}
            child{ node [arn_r] {20} 
            	child {node [arn_x] {}}
            	child {node [arn_x] {}}        	
          	}
            child{ node [arn_r] {15}
            	child{node [arn_x] {}}
            	child{node [arn_x]{}}
            }
   };
\end{tikzpicture}
\\
\vspace{2.5mm}\\
\begin{tabular}{|c|c|c|c|c|c|c|c|c|c|c|}
\hline K & J & I & B & Q & C & G & D & F & E & H \\ 
\hline 4 & 5 & 12 & 9 & 8 & 20 & 15 & 18 & 10 & 13 & 11 \\ 
\hline 
\end{tabular} \\
\vspace{10mm}\\
\indent Apr\`es l'ajout de 10:\\
\begin{tikzpicture}[->,>=stealth',level/.style={sibling distance = 5cm/#1,
  level distance = 1.5cm}] 
\node [arn_r] {4}
    child{ node [arn_r] {5} 
            child{ node [arn_r] {9} 
            	child{ node [arn_r] {18}} 
				child{ node [arn_r] {10}}
            }
            child{ node [arn_r] {8}
							child{ node [arn_r] {13}}
							child{ node [arn_r] {11}}
            }                            
    }
    child{ node [arn_r] {10}
            child{ node [arn_r] {12} 
            	child {node [arn_r] {20}}
            	child {node [arn_x] {}}        	
          	}
            child{ node [arn_r] {15}
            	child{node [arn_x] {}}
            	child{node [arn_x]{}}
            }
   };
\end{tikzpicture}
\\
\vspace{2.5mm}\\
\begin{tabular}{|c|c|c|c|c|c|c|c|c|c|c|}
\hline K & J & I & B & Q & C & G & D & F & E & H \\ 
\hline 4 & 5 & 12 & 9 & 8 & 20 & 15 & 18 & 10 & 13 & 11 \\ 
\hline 
\end{tabular} \\
\vspace{5mm}\\
\indent c- algorithme d'ajout dans un tas:\\
algorithme procedure ajout\\
\indent parametres globaux\\
\indent \indent t\_hierar tas\\
 
\end{document}