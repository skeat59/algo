 \documentclass [11pt]{report}

\usepackage{fancyhdr}
\usepackage [french]{babel}

\usepackage[utf8]{inputenc}
\usepackage[T1]{fontenc}
\usepackage{textcomp}
\usepackage{graphicx}
\usepackage{titlepic}

\usepackage{listings}
\usepackage{minitoc}
\usepackage{footmisc}
\usepackage{color}
\usepackage{graphicx}

\usepackage{eso-pic}
\pagestyle{fancy}	
\begin{document}
/* t\_eltsFile */ \\
\indent \indent t\_File = \textuparrow t\_xxx \\
\indent \indent t\_xxx = enregistrement \\
\indent \indent \indent t\_eltFile elt \\
\indent \indent \indent t\_File tete, queue, suiv, prec\\
\indent \indent fin enregistrement t\_xxx\\
\\
\\
algorithme fonction File\_Vide: t\_File \\
debut\\
\indent retourne(NUL)\\
fin algorithme fonction File\_Vide\\
\\
algorithme fonction Est\_Vide: booleen\\
\indent parametres locaux \\
\indent \indent t\_File F
debut \\
\indent retourne ( NUL = p )\\
fin algorithme fonction Est\_Vide\\
\\
			algorithme fonction enfiler: t\_File\\
\indent 		parametres locaux \\
\indent \indent 	t\_File F \\
\indent \indent 	t\_eltFile e\\
\indent 		variables \\
\indent \indent 	t\_pile tmp\\
			debut\\
\indent			allouer (tmp)\\
\indent 		tmp\textuparrow.elt <- e\\
\indent 		queue\textuparrow.suiv <- tmp\\
\indent 		queue <- queue\textuparrow.suiv\\
\indent 		retourne (tete)\\
			fin algorithme fonction enfiler\\
\\
\newpage
						algorihtme fonction defiler: t\_File \\
\indent				 		parametres globaux\\
\indent \indent			 		t\_File F \\
\indent				 		variables \\
\indent \indent 				t\_File tmp\\
\indent \indent					t\_Fileelt e
\indent 				debut \\
\indent \indent 			si NUL <> p alors\\
\indent \indent \indent 		allouer (tmp)
\indent \indent \indent 		tmp <- tete\textuparrow.suiv \\
\indent \indent \indent 		
\indent \indent \indent 		liberer (tete)\\
\indent \indent 			sinon\\
\indent \indent \indent 		tmp <-NUL\\
\indent\indent 				fin si\\
\indent \indent 			retourne (tmp)\\
\indent 				fin algorithme fonction defiler\\

\indent algorithme fonction destruction: t\_File\\
\indent \indent  parametres globaux\\
\indent \indent \indent t\_File F\\
\indent debut \\
\indent \indent si NUL <> p\textuparrow.suiv alors\\
\indent \indent \indent tant que tete\textuparrow.suiv <> NUL \\
\indent \indent \indent \indent tmp <-tete\textuparrow.suiv \\
\indent \indent \indent \indent liberer (tete)\\
\indent \indent \indent \indent tete <-tmp\\
\indent \indent \indent fin tant que \\
\indent \indent \indent retourne (tete)\\
\indent \indent sinon\\
\indent \indent \indent tete <- NUL\\
\indent \indent fin si\\
\indent fin algorithme fonction destruction\\
\newpage
exercice 2.4\\
\indent 1) La liste est construite a l'envers.\\

\end{document}